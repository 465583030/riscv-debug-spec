\chapter{JTAG Debug Transport Module} \label{jtagdtm}

This Debug Transport Module is based around a normal JTAG Test Access Port
(TAP).  The JTAG TAP allows access to arbitrary JTAG registers by first
selecting one using the JTAG instruction register (IR), and then accessing it
through the JTAG data register (DR).

\section{Background}

JTAG refers to IEEE Std 1149.1-2013. It is a standard that defines test logic
that can be included in an integrated circuit to test the interconnections
between integrated circuits, test the integrated circuit itself, and observe or
modify circuit activity during the component’s normal operation.
This specification uses the latter functionality.
The JTAG standard defines a Test Access Port (TAP) that
can be used to read and write a few custom registers, which can be used to
communicate with debug hardware in a component.

\section{JTAG Connector}

Every target's JTAG connector seems to have its own pinout. To make it easy to
acquire debug hardware, this spec recommends a connector that is compatible
with the Cortex Debug Connector, as described below.

The connector is a .05"-spaced, gold-plated male header with .016" thick
hardened copper or beryllium bronze square posts (SAMTEC FTSH-105 or
equivalent). Female connectors are compatible \SI{20}{\micro\metre} gold
connectors in order to prevent oxide build-up on tin connectors.

Viewing the male header from above (the pins pointing at your eye), a target's
connector looks as it does in Table~\ref{tab:header}. The function of each pin
is described in Table~\ref{tab:pinout}.

This header does not include the optional JTAG TRST signal. If an implementation
requires TRST to be driven at Power-On, this must be handled at the
PCB level and is not exported to this header.

TODO: If the above is too harsh of a requirement, we should pick a different header to
standardize on, such as the also very common AVR JTAG Header.

\begin{table}[htp]
    \centering
    \caption{JTAG Connector Diagram}
    \label{tab:header}
    \begin{tabulary}{\textwidth}{|r|r|r|l|}
        \hline
        VCC & 1 & 2 & TMS \\
        \hline
        GND & 3 & 4 & TCK \\
        \hline
        GND & 5 & 6 & TDO \\
        \hline
        KEY & 7 & 8 & TDI \\
        \hline
        N/C & 9 & 10 & RESET \\
        \hline
    \end{tabulary}
\end{table}

\begin{table}[htp]
    \centering
    \caption{JTAG Connector Pinout}
    \label{tab:pinout}
    \begin{tabulary}{\textwidth}{|r|l|L|}
        \hline
        1 & VCC & Power provided by the target, which may be used to power the
        debug adapter. Must be able to source at least 25mA. This signal also
        serves as the reference voltage for logic high.

        This pin must be clearly marked in both male and female headers.\\
        \hline
        2 & TMS & JTAG TMS signal, driven by debug adapter. \\
        \hline
        3 & GND & Target ground. \\
        \hline
        4 & TCK & JTAG TCK signal, driven by the debug adapter. \\
        \hline
        5 & GND & Target ground. \\
        \hline
        6 & TDO & JTAG TDO signal, driven by the target. \\
        \hline
        7 & KEY & This pin should be clipped in male connectors, and plugged in
        female connectors. Electrically it must not be connected. \\
        \hline
        8 & TDI & JTAG TDI signal, driven by the debug adapter.

        This pin may be used by a target to sense a debugger at reset by weakly
        pulling this signal high during a brief detection period at reset.
        Debuggers should drive TDI low when the interface is idle. \\
        \hline
        9 & N/C & Not connected in either target or debug adapter. May be used
        in future specs. \\
        \hline
        10 & RESET & Reset signal, driven by the debug adapter. This may be
        active low or active high, depending on the target's requirements. A
        debug adapter must accommodate either option. Asserting reset should
        reset any RISC-V cores as well as any other peripherals on the PCB.
        It should not reset the debug logic.
        If not implemented in a target, this pin must not be connected. \\
        \hline
    \end{tabulary}
\end{table}

Target connectors may be shrouded. In that case the key slot should be next to
pin 5. Female headers should have a matching key.

Debug adapters should be tagged or marked with their isolation voltage
threshold (i.e. unisolated, 250V, etc.).

All debug adapter pins other than GND should be current-limited to 20mA.

\section{JTAG Registers}

JTAG TAPs used as a DTM must have an IR of at least 5 bits.
When the TAP is reset, IR must default to
00001, selecting the IDCODE instruction. A full list of JTAG registers along
with their encoding is in Table~\ref{table:jtag_registers}. The only regular
JTAG registers a debugger might use are BYPASS and IDCODE, but this
specification leaves IR space for many other standard JTAG instructions.
Unimplemented instructions must select the BYPASS register.

\input{jtag_registers.tex}
