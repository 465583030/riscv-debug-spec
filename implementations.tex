\chapter{Hardware Implementations}
\label{sec:implementations}

Below are two possible implementations. A designer could choose one, mix and
match, or come up with their own design.

\section{Abstract Command Based}

Halting happens by stalling the processor execution pipeline.

Muxes on the register file(s) allow for accessing GPRs and CSRs
using the Access Register abstract command.

\section{Execution Based}

This implementation only implements the Access Register abstract commannd
for GPRs on a halted hart, and relies on the Program Buffer for all other
operations.

This method uses the processor's existing pipeline
and ability to execute from arbitrary memory locations to avoid
modifications to a processors datapath.
When \Fhaltreq is set, the Debug Module raises a special interrupt
to the selected hart. This interrupt causes the
hart to enter Debug Mode and jump to a 
memory region that is serviced by the DM and execute a 'park loop'.
When taking this exception, \Rpc is saved to \Rdpc.
In
the park loop the hart writes its \Rmhartid to a memory location within the Debug
Module to indicate that it is halted.
To allow the DM to individually control one out of several
halted harts, each hart polls a specific memory location or bit in a \Rdscratch
CSR to determine whether the debugger wants it to continue.

\Rdatazero etc. are mapped into regular memory at an address relative to \Rzero
with only a 12-bit {\tt imm}. The exact address is an implementation
detail that a debugger must not rely on.

To implement the abstract 32-bit GPR access instructions, the debugger causes
the hart to  execute {\tt lw <gpr>, 0x400(zero)} or {\tt sw0x400(zero), <gpr>}.
64- and 128-bit accesses use {\tt ld}/{\tt sd} and {\tt
lq}/{\tt sq} respectively.

To execute the Program Buffer, the debugger causes the hart to execute
{\tt j dm\_program\_buffer}. When {\tt ebreak} is executed (indicating the end of the
Program Buffer code) the hart jumps back to its park loop. If an exception is
encountered, the hart jumps to its debug exception address, which
writes to an address in the Debug Module which indicates exception, then
contains a jump back to the hart's park loop.
The DM infers from the write that there was an exception, and sets \Fcmderr appropriately.

To resume execution, the debug module causes the core to execute a {\tt dret}.
When {\tt dret} is executed, \Rpc is restored from \Rdpc and normal execution resumes at the
privilege set by \Fprv.
