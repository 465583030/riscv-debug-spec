\chapter{RISC-V Debug}
\label{sec:core_debug}

Modifications to the RISC-V core to support debug are kept to a minimum.  There
is a special execution mode (Debug Mode) and a few extra CSRs. The DM takes care
of the rest.

\section{Debug Mode} \label{debugmode}

Debug Mode is a special processor mode used only when the core is halted for
external debugging. How Debug Mode is entered is implementation-specific.

\begin{steps}{When executing code from the Program Buffer, the processor stays
    in Debug Mode and the following apply:}
\item All operations happen in machine mode.
\item \Fmprv in \Rmstatus is ignored.
\item All interrupts are masked.
\item Exceptions don't update any registers.  That includes {\tt cause}, {\tt
    epc}, {\tt badaddr}, {\tt dpc}, and \Rmstatus. They do end execution of the
    Program Buffer.
\item No action is taken if a trigger matches.
\item Trace is disabled.
\item Counters may be stopped, depending on \Fstopcount in \Rdcsr.
\item Timers may be stopped, depending on \Fstoptime in \Rdcsr.
\item The {\tt wfi} instruction acts as a {\tt nop}.
\item Almost all instructions that change the privilege level have undefined
    behavior.  This includes {\tt ecall}, {\tt mret}, {\tt hret}, {\tt sret},
    and {\tt uret}.  (To change the privilege level, the debugger can write
    \Fprv in \Rdcsr). The only exception is {\tt ebreak}. When that is executed
    in Debug Mode, it halts the processor again but without updating \Rdpc or \Rdcsr.
\end{steps}

\section{Load-Reserved/Store-Conditional Instructions}

The reservation registered by an {\tt lr} instruction on a memory address may
be lost when entering Debug Mode or while in Debug Mode.  This means that there
may be no forward progress if Debug Mode is entered between {\tt lr} and {\tt
sc} pairs.

\section{Reset}

If the halt signal is asserted when a core comes out of reset, the core must
enter Debug Mode before executing any instructions, but after performing any
initialization that would usually happen before the first instruction is
executed.

\section{Core Debug Registers} \label{debreg}

The supported Core Debug Registers must be implemented for each hart that can
be debugged.

\input{core_registers.tex}

\section{Virtual Debug Registers} \label{virtreg}

Virtual debug registers are a requirement on the debugger SW/interface,
not on the Core designer.

<registers name="Virtual Core Debug Registers" prefix="VIRT_">
  Users of the debugger shouldn't need to know about the core debug registers,
  but may want to change things affected by them. 
     A virtual register is one that doesn't exist directly in the hardware, but that the debugger exposes as if it does. 
    
     <register name="Privilege Level" short="priv" address="virtual">
        User can read this register to  inspect the privilege level that
        the hart was running in when the hart halted.
        User can write this register to change the privilege level that
        the hart will run in when it resumes.

        \begin{table}
        \centering
        \caption{Privilege Level Encoding}
        \label{tab:privlevel}
        \begin{tabular}{|r|l|}
        \hline
        Encoding &amp; Privilege Level \\
        \hline
        0 &amp; User/Application \\
        1 &amp; Supervisor \\
        2 &amp; Hypervisor \\
        3 &amp; Machine \\
        \hline
        \end{tabular}
        \end{table}

        <field name="prv" bits="1:0" access="R/W" reset="0">
            Contains the privilege level the hart was operating in when Debug
            Mode was entered. The encoding is described in Table
            \ref{tab:privlevel}. A user can write this value to change the
            hart's privilege level when exiting Debug Mode.
        </field>
    </register>
</registers>

